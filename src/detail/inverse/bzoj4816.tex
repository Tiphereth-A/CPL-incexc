\begin{frame}[fragile,allowframebreaks]{BZOJ 4816 {[}SDOI2017{]} 数字表格}
	\label{incexc:example:bzoj4816}
	\textbf{题目描述}

	Doris 刚刚学习了 fibonacci 数列. 用 \(f_i\) 表示数列的第 \(i\) 项, 那么

	\[
		f_0=0,f_1=1
	\]

	\[
		f_n=f_{n-1}+f_{n-2},n\geq 2
	\]

	Doris 用老师的超级计算机生成了一个 \(n\times m\) 的表格, 第 \(i\) 行第 \(j\) 列的格子中的数是 \(f_{\gcd(i,j)}\), 其中 \(\gcd(i,j)\) 表示 \(i,j\) 的最大公约数. Doris 的表格中共有 \(n\times m\) 个数, 她想知道这些数的乘积是多少. 答案对 \(10^9+7\) 取模

	\textbf{输入格式}

	有多组测试数据

	输入的第一行是一个整数 \(T\), 表示测试数据的组数

	接下来 \(T\) 行, 每行两个整数 \(n, m\)

	\(1 \leq T\leq 10^3\), \(1\leq n,m\leq 10^6\)

	\textbf{输出格式}

	输出\(T\)行, 第 \(i\) 行的数是第 \(i\) 组数据的结果

	\textbf{样例输入}

	\includecode[common]{bzoj4816.in}

	\textbf{样例输出}

	\includecode[common]{bzoj4816.ans}
\end{frame}


\begin{frame}{题解}
	不妨设 \(n\geqslant m\), 则

	\pause
	\[
		\begin{aligned}
			\prod_{i=1}^n\prod_{j=1}^mf_{(i,j)} & =\exp\log\left(\prod_{d=1}^n\prod_{i=1}^{\lfloor\frac{n}{d}\rfloor}\prod_{j=1}^{\lfloor\frac{m}{d}\rfloor}f_d^{[(i,j)=1]}\right)             \\
			                                    & =\exp\sum_{d=1}^n\left((\log f_d)\sum_{i=1}^{\lfloor\frac{n}{d}\rfloor}\sum_{j=1}^{\lfloor\frac{m}{d}\rfloor}[(i,j)=1]\right)                \\
			                                    & =\exp\sum_{d=1}^n\left((\log f_d)\sum_{i=1}^{\lfloor\frac{n}{d}\rfloor}\sum_{j=1}^{\lfloor\frac{m}{d}\rfloor}\sum_{e\mid (i,j)}\mu(e)\right) \\
			                                    & =\exp\sum_{d=1}^n\left((\log f_d)\sum_{e=1}^{\lfloor\frac{n}{d}\rfloor}\mu(e)\lfloor\frac{n}{de}\rfloor\lfloor\frac{m}{de}\rfloor\right)     \\
			                                    & \xlongequal{D=de}\prod_{D=1}^n\bigg(\prod_{d\mid D}f_d^{\mu(\frac{D}{d})}\bigg)^{\lfloor\frac{n}{D}\rfloor\lfloor\frac{m}{D}\rfloor}
		\end{aligned}
	\]
\end{frame}


\begin{frame}{题解}
	其中 \(\prod_{d\mid D}f_d^{\mu(\frac{D}{d})}\) 部分可以 \(O(n\log n)\) 求出, 之后就是整除分块了

	\pause
	\textbf{时间复杂度} \(O(n\log n)\)
\end{frame}


\begin{frame}[fragile,allowframebreaks]{参考实现}
	\includecode[c++]{bzoj4816.cpp}
\end{frame}
